\section{\href{http://mirrors.ctan.org/graphics/pgf/base/doc/pgfmanual.pdf}{Tikz}}

\subsection*{Picture Settings}
\begin{minipage}{6cm}
At beginning:\\
\begin{lstlisting}
\begin{tikzpicture}[<setting>*]
\end{lstlisting}
... or mid-course:\\
\begin{lstlisting}
\tcbset{<setting>*}
\end{lstlisting}
\end{minipage}

Possible settings include: 
\begin{itemize}
    \item remember picture
    \item overlay
    \item \mathrm{[x|y] = (coord)}
\end{itemize}
\ \\

\subsection*{Styles \& Aliases}
\begin{code}{6}
\begin{lstlisting}
\begin{tikzpicture}
    [<alias>/.style={<style>*}]
\end{lstlisting}
\end{code} \\

Possible styles include: \\
\begin{tabularx}{5cm}{l l}
    style & example val \\
    \hline \\
    inner sep & 2pt\\
    text & green \\
    font & small \\
    anchor & west \\
    align & \\
    pos & .5 \\
    decorate & --\\
    decoration & snake \\
    fill & green \\
    color & green\\
    draw & green \\
    rounded corners & \\
\end{tabularx} \\

\subsection*{Nodes}
\begin{code}{6.5}
\begin{lstlisting}
\begin{node}[<setting>*](<title>)
\end{lstlisting}
\end{code} \\
\begin{tabularx}{4cm}{r l}
    setting & example \\
    \hline
    anchor & west \\
    align & right|justify \\
    text width & .85cm \\
    pos & .5 \\
    below of & my\_node \\
    text depth & 5.7cm \\
    label & \{xshift=.5mm\} \\
    shape & rectangle callout \\ 
\end{tabularx} \\


\subsection*{Paths \& Draws}
\begin{code}{6.5}
\begin{lstlisting}
\[fill]draw[<setting>*]
  (<from>) -- (<to>) [-- (<to>)]*;
\end{lstlisting}
\end{code}
<from> \& <to> are coordinates (see below). Joiners like (<a>) -- (<to>) can also connect at right angles, like: (<from>) -| (<to>) or (<from>) |- (<to>). Possible settings include: \\
\begin{tabularx}{4cm}{l l}
    setting & desc. \\
    \hline
    -> & unidirect. arrow \\
    <-> & double arrow \\
    >={<head>[option]} & arrow-head \\ 
\end{tabularx} \\
Arrow <head>s include permutations of:
\begin{itemize}
    \item \{stealth\}
    \item \{stealth\textquotesingle\}
    \item \{open triangle 45\}
    \item \{latex reversed double\}
\end{itemize}
... and options such as: 
\begin{itemize}
    \item width
    \item length
\end{itemize}
\begin{code}{6.5}
Another method:
\begin{lstlisting}
\path[<setting>*]
  (<from>) edge[<setting>*] (<to>);
\end{lstlisting}
... where settings $\in$: \\
\begin{itemize}
    \item out|in=<degrees>
    \item bend right|left=<degrees>
\end{itemize}
\end{code}

\ \\

\subsection*{Coordinates}
Referencing coordinates: \\
\begin{tabularx}{4cm}{l l}
    type & example \\
    \hline 
    named & (my\_coord) \\
    anchor & (my\_node.north east) \\
    absolute & (1,-3) \\
    relative & ++(2,2) \\
    tween & (node\_a)!.25!(node\_b) \\
    math & \$(my\_name) + (1,3)\$ \\
    intersect & (<coor\_a> -| <coord\_b>)
\end{tabularx}

Declaring / naming coordinates: \\
\begin{tabular}{l}
\begin{lstlisting}
\coordinate()
\end{lstlisting} \\
\begin{lstlisting}
\draw(a)--(b) coordinate(<coord>);
\end{lstlisting} \\
\end{tabular}


\ \\